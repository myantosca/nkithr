\documentclass[11pt,epsf]{article}
\usepackage{amssymb,amsmath,amsthm,amsfonts,mathrsfs,color}
\usepackage{epsfig}
\usepackage{latexsym}
\usepackage{verbatim}
\usepackage{setspace}
\usepackage{algorithm}
\usepackage[noend]{algorithmic}
\usepackage{algorithmicext}
\usepackage{ifthen}
\usepackage{graphicx}
\usepackage{url}
\usepackage{hyperref}
\usepackage[utf8]{luainputenc}
\usepackage[bibencoding=utf8,backend=biber]{biblatex}
\addbibresource{cosc6326-pa1-michael-yantosca.bib}
\usepackage{fancyhdr}
\pagestyle{fancy}
\lhead{{\footnotesize{COSC6326 PA 1}}}
\rhead{{\footnotesize{Michael Yantosca}}}

\usepackage{longtable}
\usepackage{pgfplots}
\usepackage{pgfplotstable}
\usepgfplotslibrary{external}
\usepgfplotslibrary{statistics}
\usepgfplotslibrary{groupplots}
\usetikzlibrary{pgfplots.groupplots, external}
\tikzexternalize[]
\pgfplotsset{
  tick label style={font=\footnotesize},
  label style={font=\small},
  legend style={font=\small},
  compat=newest
}
\pgfplotstableset{
  col sep=comma,
  begin table=\begin{longtable},
  end table=\end{longtable},
  every head row/.append style={after row=\endhead}
}

\newtheorem{fact}{Fact}
\newtheorem{theorem}{Theorem}
\newtheorem{lemma}{Lemma}
\newtheorem{claim}{Claim}
\newtheorem{remark}{Remark}
\newtheorem{definition}{Definition}
\newtheorem{corollary}{Corollary}
\newtheorem{proposition}{Proposition}
\newtheorem{example}{Example}
\newtheorem{observation}{Observation}
\newtheorem{exercise}{Exercise}
\newtheorem{statement}{Statement}
\newtheorem{problem}{Problem}

\newcommand{\TODO}[0]{\textbf{\color{red}{TODO}}}

% \linregplots{title}{prefix}{suffix}{x}{y}
\newcommand{\linregplots}[5]{
  \nextgroupplot[title=#1]
  \addplot [red, only marks, mark size=0.5] table [x=#4, y=#5] {#2.k1#3.log};
  \addplot [red, no markers] table [x=#4,y={create col/linear regression={y=#5}}] {#2.k1#3.log};
  \addplot [blue, only marks, mark size=0.5] table [x=#4, y=#5] {#2.k2#3.log};
  \addplot [blue, no markers] table [x=#4,y={create col/linear regression={y=#5}}] {#2.k2#3.log};
  \addplot [green, only marks, mark size=0.5] table [x=#4, y=#5] {#2.k4#3.log};
  \addplot [green, no markers] table [x=#4,y={create col/linear regression={y=#5}}] {#2.k4#3.log};
  \addplot [orange, only marks, mark size=0.5] table [x=#4, y=#5] {#2.k8#3.log};
  \addplot [orange, no markers] table [x=#4,y={create col/linear regression={y=#5}}] {#2.k8#3.log};
  \addplot [purple, only marks, mark size=0.5] table [x=#4, y=#5] {#2.k16#3.log};
  \addplot [purple, no markers] table [x=#4,y={create col/linear regression={y=#5}}] {#2.k16#3.log};
  \addplot [brown, only marks, mark size=0.5] table [x=#4, y=#5] {#2.k32#3.log};
  \addplot [brown, no markers] table [x=#4,y={create col/linear regression={y=#5}}] {#2.k32#3.log};
}

\date{}
\title{COSC6326 Programming Assignment 1}
\author{Michael Yantosca}
\begin{document}
\maketitle
\tableofcontents

\section{Introduction}{
  \paragraph{}{
    As problem sizes continue to scale, it would behoove algorithmic researchers to explore
    not only more distributed and decentralized approaches but especially randomized versions
    of these approaches. One particular information retrieval problem, namely the selection
    of the $i$th element among a large set of $n$ inherently ordered elements provides a basic
    but effective overview of some of the challenges at hand and the benefits that a randomized,
    distributed approach provides along with some of the insights that implementing such a
    system may offer, not only in the field of randomized algorithms, but in the field of
    deterministic algorithms, as well.
  }
}

\section{Analysis}{
  \subsection{\texttt{nkith}}{
    \paragraph{}{
      \begin{algorithm}
        \footnotesize
        \caption{\textsc{MP-Ith-Select}, Distributed Algorithm for Selecting the \emph{i}th Element of \emph{n} Numbers}
        \begin{algorithmic}
          \label{alg:nkith}
          \REQUIRE{$G$, a PRNG}\autocite{mt19337}
          \REQUIRE{$n$, global population size}
          \REQUIRE{$r$, node rank}
          \REQUIRE{$k$, number of nodes}
          \REQUIRE{
            \[
              m \gets n \div k +
              \begin{cases}
                1, & \text{ if } n \text{ mod } k > r \\
                0, & \text{otherwise}
              \end{cases},
              \text{ local population size}
            \]
          }
          \REQUIRE{$p \gets 0$, round-robin pivot selection counter}
          \REQUIRE{$b_l \gets 0$, live local population lower bound}
          \REQUIRE{$b_u \gets m$, live local population upper bound}
          \REQUIRE{$|S_{1,r}| \gets 0$, local cardinality counter for elements below the pivot}
          \REQUIRE{$|P_r| \gets 0$, local cardinality counter for elements less than or equal to the pivot}
          \REQUIRE{$|S_1| \gets 0$, global cardinality counter for elements below the pivot}
          \REQUIRE{$|P| \gets 0$, global cardinality counter for elements less than or equal to the pivot}
          \STATE{Seed $G$ with the current epoch time.}
          \FOR{$j = 0 \text{ to } m - 1$}{
            \STATE{$M[j] = \textsc{Generate}(G)$}
          }\ENDFOR
          \STATE{\textsc{Qsort}($M$)\autocite{qsort}}
          \WHILE{\neg ($|S_1| < i$ and $|S_1| + |P| \geq i$)}{
            \STATE{$c_r \gets M[b_l + (b_u - b_l) / 2]$}
            \STATE{All nodes send $c_r$ to the root node.}
            \IF{$r = 0$}{
              \STATE{$c = c_p$}
              \STATE{$p = (p + 1) \text{ mod } k$}
              \STATE{Broadcast $c$ to all nodes.}
            }\ENDIF
            \STATE{$|S_{1,r}| \gets b_l$}
            \WHILE{$M[|S_{1,r}|] < c$ and $|S_1| < b_u$}{
              \STATE{$|S_{1,r}| = |S_{1,r}| + 1$}
            }\ENDWHILE
            \STATE{$|P_r| \gets |S_{1,r}|$}
            \WHILE{$M[|P_r|] == c$ and $|P_r| < b_u$}{
              \STATE{$|P_r| = |P_r| + 1$}
            }\ENDWHILE
            \STATE{All nodes send $|S_{1,r}|$ and $|P_r|$ to the root node.}
            \IF{$r = 0$}{
              \STATE{$|S_1| \gets \sum_{r = 0}^{k-1} |S_{1,r}|$}
              \STATE{$|P| \gets \sum_{r = 0}^{k-1} |P_r|$}
              \STATE{Broadcast $|S_1|$ and $|P|$ to all nodes.}
            }\ENDIF
            \IF{$|S_1| > i - 1$}{
              \STATE{$b_u \gets |S_{1,r}|$}
            }\ELSIF{$|P| < i$}{
              \STATE{$b_l \gets |P_r|$}
            }\ENDIF
          }\ENDWHILE
          \STATE{The root node announces $c$ as the $i$th element.}
        \end{algorithmic}
      \end{algorithm}
    }
    \paragraph{}{
      A full analysis of Algorithm~\ref{alg:nkith} is omitted since it does not faithfully
      reproduce the sequential algorithm. Indeed, a proper analysis is beyond the scope of
      this assignment since the implementation of \textsc{Qsort} is dependent on the
      particular POSIX implementation on which the code is run. The pseudocode is included
      for reference and completeness since this first attempt served as the progenitor for
      the correct implementation in Algorithm~\ref{alg:nkithr}.
    }
  }

  \subsection{\texttt{nkithr}}{
    \paragraph{}{
      \begin{algorithm}
        \footnotesize
        \caption{\textsc{Random-Pivot-Select}, Distributed Algorithm for Selecting a Pivot Candidate}
        \begin{algorithmic}
          \label{alg:random-pivot-select}
          \REQUIRE{$k$, number of nodes}
          \REQUIRE{$b_l$, live local population lower bound}
          \REQUIRE{$b_u$, live local population upper bound}
          \STATE{$G_{cand} \gets$ uniform distribution over $[b_l, b_u)$}\autocite{uniformintdist}
          \STATE{$c_r \gets M[\textsc{Generate}(G_{cand})]$}
          \STATE{$w_r \gets b_u - b_l$}
          \STATE{All nodes send $c_r$ and $w_r$ to the root node.}
          \IF{$r = 0$}{
            \STATE{$w \gets \sum_{r=0}^{k-1} w_r$}
            \STATE{$G_{pivot} \gets$ distribution over $[0,k)$ weighted by $w_r/w$.}\autocite{discretedist}
            \STATE{$p = \textsc{Generate}(G_{pivot})$}
            \STATE{$c = c_p$}
            \STATE{Broadcast $c$ to all nodes.}
          }\ENDIF
        \end{algorithmic}
      \end{algorithm}
    }

    \paragraph{}{
      \begin{algorithm}
        \footnotesize
        \caption{\textsc{MP-Ith-Select-Revised}, Distributed Algorithm for Selecting the \emph{i}th Element of \emph{n} Numbers}
        \begin{algorithmic}
          \label{alg:nkithr}
          \REQUIRE{$G_{popl}$, a population PRNG}\autocite{mt19337}
          \REQUIRE{$n$, global population size}
          \REQUIRE{$r$, node rank}
          \REQUIRE{$k$, number of nodes}
          \REQUIRE{$c$, the current pivot}
          \REQUIRE{
            \[
              m \gets n \div k +
              \begin{cases}
                1, & \text{ if } n \text{ mod } k > r \\
                0, & \text{otherwise}
              \end{cases},
              \text{ local population size}
            \]
          }
          \REQUIRE{$b_l \gets 0$, live local population lower bound}
          \REQUIRE{$b_u \gets m$, live local population upper bound}
          \REQUIRE{$|S_1| \gets 0$, global cardinality for elements below the pivot}
          \REQUIRE{$|P| \gets 0$, global cardinality for elements less than or equal to the pivot}
          \STATE{Seed $G_{popl}$ with the current epoch time.}
          \FOR{$j = 0 \text{ to } m - 1$}{
            \STATE{$M[j] = \textsc{Generate}(G_{popl})$}
          }\ENDFOR
          \WHILE{\neg ($|S_1| < i$ and $|S_1| + |P| \geq i$)}{
            \STATE{$\textsc{Random-Pivot-Select}(b_l, b_u)$ \COMMENT{Algorithm~\ref{alg:random-pivot-select}}}
            \STATE{$s_1 \gets b_l$}
            \STATE{$s_p \gets 0$}
            \STATE{$s_2 \gets b_u$}
            \STATE{$j \gets b_l$}
            \FOR{$j = b_l \text{ to } b_u - 1$}{
              \IF{$M[j] < c$}{
                \STATE{$Q[s_1] \gets M[j]$}
                \STATE{$s_1 = s_1 + 1$}
              }\ELSIF{$M[j] > c$}{
                \STATE{$s_2 = s_2 -1$}
                \STATE{$Q[s_2] \gets M[j]$}
              }\ELSE{
                \STATE{$s_p \gets s_p + 1$}
              }\ENDIF
            }\ENDFOR
            \STATE{$s_p \gets s_p + s_1$}
            \FOR{$j = s_1 \text{ to } s_p$}{
              \STATE{$Q[j] \gets c$}
            }\ENDFOR
            \IF{$b_l < b_u$}{
              \STATE{Copy $Q[b_l, b_u-1]$ to $M[b_l, b_u-1]$.}
            }\ENDIF
            \STATE{All nodes send $s_1$ and $s_p$ to the root node.}
            \IF{$r = 0$}{
              \STATE{$|S_1| \gets \sum_{r = 0}^{k-1} s_{1,r}$}
              \STATE{$|P| \gets \sum_{r = 0}^{k-1} s_{p,r}$}
              \STATE{Broadcast $|S_1|$ and $|P|$ to all nodes.}
            }\ENDIF
            \IF{$|S_1| > i - 1$}{
              \STATE{$b_u \gets s_1$}
            }\ELSIF{$|P| < i$}{
              \STATE{$b_l \gets s_p$}
            }\ENDIF
          }\ENDWHILE
          \STATE{The root node announces $c$ as the $i$th element.}
        \end{algorithmic}
      \end{algorithm}
    }
    \begin{theorem}
      \label{thm:nkithr}
      Algorithm~\ref{alg:nkithr} determines the $i$th element of a set of $n$ numbers
      distributed over $k$ nodes with time complexity $O(\log n)$ rounds and message complexity
      $O(k \log n)$ on average.
    \end{theorem}
    \begin{lemma}
      \label{lem:nkithr-t}
      Algorithm~\ref{alg:nkithr} determines the $i$th element of a set of $n$ numbers
      distributed over $k$ nodes with time complexity $O(\log n)$ rounds on average.
    \end{lemma}
    \begin{proof}
      The proof follows the sketch outlined by the proof for the time
      complexity of Randomized Quicksort given in the sequential algorithms
      text by Pandurangan\autocite[168]{ALG}. For simplicity, we limit the proof to the
      case where the $i$th element is the median, i.e., $\lfloor \frac{n}{2} \rfloor$
      among $n$ numbers and the pivot selection is random and independent of the
      population distribution.
      \paragraph{}{
        The first order of business is to simplify the proof so that we can
        speak in global terms without having to perform complicated
        if not intractable calculations on local probabilities.
      }
      \begin{lemma}
        \label{lem:goodpivot}
        The pivot selected by Algorithm~\ref{alg:random-pivot-select} chooses any
        member of the live (i.e., considered) population $n_j$ in the $j$th iteration
        with uniform probability $\frac{1}{n_j}$.
      \end{lemma}
      \begin{proof}
        The local candidate is selected from the local live population of size $m_j = b_{u,j} - b_{l,j}$
        with $\frac{1}{m_j}$ probability. The corresponding weight sent to the root
        node with the candidate is $m_j$. The global live population is faithfully
        recorded at the root as $n_j = \sum_{h=0}^{k-1} m_{h,j}$. The weights are normalized
        against $n_j$ so that each becomes respectively $\frac{m_{h,j}}{n_j}$.
        The node identifier is selected according to a discrete distribution weighted
        with the aforementioned weights. The probability that any given element
        will be selected as the pivot is given as follows:
        \begin{align*}
          P(\text{element selected globally})
          &= P(\text{node selected globally}|\text{element selected locally})\\
          &\phantom{= }\cdot P(\text{element selected locally}) \\
          &= \frac{m_{h,j}}{n_j} \cdot \frac{1}{m_{h,j}} \\
          &= \frac{1}{n_j}
        \end{align*}
        Thus, Algorithm~\ref{alg:random-pivot-select} selects each pivot in each
        iteration from among the live element population with uniform probability.
      \end{proof}
      \paragraph{}{
        By Lemma~\ref{lem:goodpivot}, we can consider the partitioning of the population
        in a global sense so long as the implementation faithfully reports the cardinalities
        of the subsets of $S_1$ (less than the pivot) and $P$ (equal to the pivot).
        We know that this is the case since the element-wise comparisons done locally
        appropriately bin the live local population into the respective subsets,
        and the cardinalities of the subsets are summed at the root and rebroadcast
        to all nodes for the purpose of updating the live population boundaries.
      }
      \paragraph{}{
        As with the proof for Theorem 8.8\autocite[168-169]{ALG}, let us consider then
        the division of outcomes into good and bad sets. Let the good outcome be that
        where the pivot is chosen in the middle third of the live global population, i.e.,
        that neither $|S_1|$ nor $|S_2|$ (greater than the pivot) exceeds $\frac{2n_j}{3}$,
        and let the bad outcome be the converse.
      }
      \paragraph{}{
        We cannot have more than $\log_{3/2} n$ good outcomes, or else the population will
        have been exhausted by the constant reduction by $\frac{2}{3}$. , the expectation for execution time is $3c\log n$,
        where $c$ is a fixed constant independent of the problem size such that
        $\log_{3/2} n < c \log n$.
      }
      \paragraph{}{
        If we consider longer paths of execution that have at most $c \log n$ good outcomes
        but are longer than that, i.e., paths of length $ac \log n$ for some $a > 1$, we can
        use the Chernoff Lower Tail Bound\autocite[323]{ALG} to provide a probabilistic
        limit on the execution time in terms of number of iterations.
        Since good outcomes occur with probability $\frac{1}{3}$, the expectation of good
        outcomes on a path of length $ac \log n$ would be $\frac{1}{3}ac \log n$.
        By choosing $\delta = 3/a$, we get a probability $P(X_{good} < c \log n) \leq \frac{1}{n^2}$.
        Consequently, with $n$ elements in play at the start, the single
        path probability becomes union bounded as $1/n$\autocite[169]{ALG}.
      }
      \paragraph{}{
        Thus, the number of iterations is on the order of $O(\log n)$ rounds with
        high probability since the greatest possible probability to exceed $ac \log n$
        rounds is $1/n$.
      }
    \end{proof}
    \begin{lemma}
      \label{lem:nkithr-m}
      Algorithm~\ref{alg:nkithr} determines the $i$th element of a set of $n$ numbers
      distributed over $k$ nodes with message complexity $O(k \log n)$ on average.
    \end{lemma}
    \begin{proof}
      By Lemma~\ref{lem:nkithr-t}, we know that Algorithm~\ref{alg:nkithr} takes
      $O(\log n)$ iterations on average. Within each iteration, the following message exchanges occur:
      \begin{enumerate}
      \item{$k - 1$ nodes send pivot candidates to the root.}
      \item{$k - 1$ nodes send pivot weights to the root.}
      \item{The root sends the pivot-elect to $k - 1$ nodes.}
      \item{$k - 1$ nodes send local cardinalities $s_1$ and $s_p$ to the root.}
      \item{The root sends global cardinalities $|S_1|$ and $|S_1| + |P|$ to $k - 1$ nodes.}
      \end{enumerate}
      With 5 exchanges of $k - 1$ messages, each iteration produces $5k - 5$. Even if we are
      nitpicky and account the 2 element local cardinality messages as 2 exchanges, we still
      only have $6k - 6$, which in either case is $O(k)$ messages per iteration. Multiplying
      by $O(\log n)$ iterations, we arrive at total message complexity of $O(k \log n)$,
      which was to be shown.
    \end{proof}
    \begin{proof}
      By Lemma~\ref{lem:nkithr-t} and Lemma~\ref{lem:nkithr-m}, we see that the two
      complexities are satisfied. Hence, Theorem~\ref{thm:nkithr} holds.
    \end{proof}
  }
}

\section{Results}{

  \subsection{Test Procedures}{
    \paragraph{}{
      Tests to validate correctness were performed locally on a dual-core laptop
      \footnote{2 x Intel(R) Core(TM) i5-6200U CPU @ 2.30GHz.} running Pop!OS.
      \footnote{An Ubuntu 18.04 variant.}
      Tests for which results were collected systematically and graphed were
      done on the UH \texttt{crill} cluster with the following parameters:
      \begin{itemize}
      \item{$n \in 2^{[10,28]}$, the total population size}
      \item{$k \in \{1,2,4,8,16,32\}$, the number of distributed nodes}
      \item{$i \in \{1, n/2, n\}$, the sought element index}
      \end{itemize}
      Each parameter combination was executed at least 10 times.
    }
    \paragraph{}{
      Several executables or versions of executables were used over the course
      of experimentation. Here are the primary variants.
    }
    \paragraph{\texttt{nkmax}}{
      A specialized implementation that used a simple local variable and stream
      processing to reduce local candidacy to one element during the generation phase.
    }
    \paragraph{\texttt{nkith}}{
      An ill-considered attempt to try and pre-sort local populations so that
      the exchange phases were simple cursor movements.
    }
    \paragraph{\texttt{nkithr}}{
      A revised version of \texttt{nkith} that properly implemented a distributed
      version of the given sequential algorithm. The first set of test runs used
      \texttt{std::default\_random\_engine}\autocite{random}. Out of a (largely unfounded)
      concern that the C++ default random engine was generating skewed pivot choices,
      the second set of test runs used \texttt{std::knuthb}\autocite{knuthb}.
      A third set of test runs was executed with a deterministic pivot selection mechanism
      similar to that used in \texttt{nkith} but with additional logic
      to skip candidates supplied by nodes whose search spaces were
      actually \emptyset.
    }
    \paragraph{\texttt{popldump}}{
      In order to empirically verify correctness, each of the above programs
      was instrumented with MPIO\autocite{MPIO} commands so that the
      entire population of a given run could be saved and read later.
      The program takes the native byte representations of the unsigned
      32-bit integers comprising the population and prints their decimal
      form with one number per line.
    }
    \paragraph{}{
      The reader is directed to the accompanying \texttt{README.md} for
      explicit usage instructions on the various programs as well as
      for a deeper explanation of implementation considerations,
      engineering tradeoffs, and known issues with the programs (mostly
      concerning the absence of defensive logic against bad command-line
      arguments). A command sequence for verifying the output correctness
      is given in the section on \texttt{popldump}. It should be noted
      that the MPI 2.1 standard\autocite{MPI21} and g++-7.2 was used for
      development since they were available through the development laptop's
      package system, but the program compiled and ran on the \texttt{crill}
      cluster with g++-5.3.0 and MPI 3.0. The code requires the C++-11 standard.
    }
  }
  \subsection{Finding the Maximum}{
    \paragraph{}{
      \begin{figure}
        \begin{tikzpicture}
          \footnotesize
          \begin{groupplot}[
              group style={
                group size=4 by 5,
                xlabels at=edge bottom,
                ylabels at=edge left,
                vertical sep=0.5in
              },
              height=1.5in,
              width=1.5in,
              xlabel=$n$,
              ylabel=$t$ (ns),
              tick label style={font=\tiny},
              label style={font=\tiny},
              legend style={font=\tiny},
              legend entries={,$k = 1$,,$k = 2$,,$k = 4$,,$k = 8$,,$k = 16$,,$k = 32$},
              legend to name=nprocs
            ]
            % nkith
            \nextgroupplot[group/empty plot]
            \linregplots{\footnotesize Total Time}{./results/nkith/nkith}{.max}{n}{t}
            \linregplots{\footnotesize Messages}{./results/nkith/nkith}{.max}{n}{m}
            \linregplots{\footnotesize Rounds}{./results/nkith/nkith}{.max}{n}{r}
            % nkithr (Knuth-B)
            \linregplots{\footnotesize Exchange Time}{./results/nkithr-knuthb/nkithr}{.max}{n}{t}
            \linregplots{}{./results/nkithr-knuthb/nkithr}{.max}{n}{T}
            \linregplots{}{./results/nkithr-knuthb/nkithr}{.max}{n}{m}
            \linregplots{}{./results/nkithr-knuthb/nkithr}{.max}{n}{r}
            % nkithr (Standard)
            \linregplots{}{./results/nkithr-stdrnd/nkithr}{.max}{n}{t}
            \linregplots{}{./results/nkithr-stdrnd/nkithr}{.max}{n}{T}
            \linregplots{}{./results/nkithr-stdrnd/nkithr}{.max}{n}{m}
            \linregplots{}{./results/nkithr-stdrnd/nkithr}{.max}{n}{r}
            % nkithr -D
            \linregplots{}{./results/nkithr-d/nkithr-d}{.max}{n}{t}
            \linregplots{}{./results/nkithr-d/nkithr-d}{.max}{n}{T}
            \linregplots{}{./results/nkithr-d/nkithr-d}{.max}{n}{m}
            \linregplots{}{./results/nkithr-d/nkithr-d}{.max}{n}{r}
            % nkmax
            \nextgroupplot[group/empty plot]
            \linregplots{}{./results/nkmax/nkmax}{}{n}{t}
            \linregplots{}{./results/nkmax/nkmax}{}{n}{m}
            \linregplots{}{./results/nkmax/nkmax}{}{n}{r}
          \end{groupplot}
          \node at (group c1r1) {\pgfplotslegendfromname{nprocs}};
          \node[anchor=south] at ($(group c2r1.north east)!0.5!(group c3r1.north west)$){\texttt{nkith}};
          \node[anchor=south] at ($(group c2r2.north east)!0.5!(group c3r2.north west)$){\texttt{nkithr} (Knuth-B)};
          \node[anchor=south] at ($(group c2r3.north east)!0.5!(group c3r3.north west)$){\texttt{nkithr} (Default)};
          \node[anchor=south] at ($(group c2r4.north east)!0.5!(group c3r4.north west)$){\texttt{nkithr} (Deterministic)};
          \node[anchor=south] at ($(group c2r5.north east)!0.5!(group c3r5.north west)$){\texttt{nkmax}};
        \end{tikzpicture}
        \caption{Empirical Complexity of Finding the Maximum}
        \label{fig:maxcplx}
      \end{figure}
    }
    \paragraph{}{
      The empirical complexity of finding the maximum is shown in Figure~\ref{fig:maxcplx}.
      The bespoke \texttt{nkmax} implementation performs best in terms of time and
      message complexity. This is to be expected since finding the max requires a
      simple counter at each node and only one round of exchange. The rejection
      occurs during generation. Consequently, the total time required for \texttt{nkmax}
      corresponds to the total time minus the exchange time for \texttt{nkithr}.
    }
    \paragraph{}{
      Interestingly, the span of the time complexity in terms of rounds corresponds
      almost exactly to the base-2 exponent of the population size for \texttt{nkithr}.
      The original \texttt{nkith} implementation with its naive deterministic pivot
      selection appears to have double the span of the revised version whether the pivot
      was chosen deterministically or at random. This is reasonable since the revised
      deterministic pivot selection depends on the calculated pivot weights to void
      ``non-participating'' nodes and avoid wasting iterations.
    }
    \paragraph{}{
      The total time complexity of \texttt{nkith} is clearly overshadowed by the
      local pre-sort time since the doubling of exchange time is not enough to
      explain the difference of \emph{half an order of magnitude} with respect to
      \texttt{nkithr}. It should be noted that early experiments with trying to do
      an insertion sort during the generation phase performed even more absymally,
      as might be expected.
    }
  }
  \subsection{Finding the Minimum}{
    \paragraph{}{
      \begin{figure}
        \begin{tikzpicture}
          \footnotesize
          \begin{groupplot}[
              group style={
                group size=4 by 4,
                xlabels at=edge bottom,
                ylabels at=edge left,
                vertical sep=0.5in
              },
              height=1.5in,
              width=1.5in,
              xlabel=$n$,
              ylabel=$t$ (ns),
              tick label style={font=\tiny},
              label style={font=\tiny},
              legend style={font=\tiny},
              legend entries={,$k = 1$,,$k = 2$,,$k = 4$,,$k = 8$,,$k = 16$,,$k = 32$},
              legend to name=nprocs
            ]
            % nkith
            \nextgroupplot[group/empty plot]
            \linregplots{Total Time}{./results/nkith/nkith}{.min}{n}{t}
            \linregplots{Messages}{./results/nkith/nkith}{.min}{n}{m}
            \linregplots{Rounds}{./results/nkith/nkith}{.min}{n}{r}
            % nkithr (Knuth-B)
            \linregplots{Exchange Time}{./results/nkithr-knuthb/nkithr}{.min}{n}{t}
            \linregplots{}{./results/nkithr-knuthb/nkithr}{.min}{n}{T}
            \linregplots{}{./results/nkithr-knuthb/nkithr}{.min}{n}{m}
            \linregplots{}{./results/nkithr-knuthb/nkithr}{.min}{n}{r}
            % nkithr (Standard)
            \linregplots{}{./results/nkithr-stdrnd/nkithr}{.min}{n}{t}
            \linregplots{}{./results/nkithr-stdrnd/nkithr}{.min}{n}{T}
            \linregplots{}{./results/nkithr-stdrnd/nkithr}{.min}{n}{m}
            \linregplots{}{./results/nkithr-stdrnd/nkithr}{.min}{n}{r}
            % nkithr -D
            \linregplots{}{./results/nkithr-d/nkithr-d}{.min}{n}{t}
            \linregplots{}{./results/nkithr-d/nkithr-d}{.min}{n}{T}
            \linregplots{}{./results/nkithr-d/nkithr-d}{.min}{n}{m}
            \linregplots{}{./results/nkithr-d/nkithr-d}{.min}{n}{r}
          \end{groupplot}
          \node at (group c1r1) {\pgfplotslegendfromname{nprocs}};
          \node[anchor=south] at ($(group c2r1.north east)!0.5!(group c3r1.north west)$){\texttt{nkith}};
          \node[anchor=south] at ($(group c2r2.north east)!0.5!(group c3r2.north west)$){\texttt{nkithr} (Knuth-B)};
          \node[anchor=south] at ($(group c2r3.north east)!0.5!(group c3r3.north west)$){\texttt{nkithr} (Default)};
          \node[anchor=south] at ($(group c2r4.north east)!0.5!(group c3r4.north west)$){\texttt{nkithr} (Deterministic)};
        \end{tikzpicture}
        \caption{Empirical Complexity of Finding the Minimum}
        \label{fig:mincplx}
      \end{figure}
    }
    \paragraph{}{
      Though not strictly part of the assignment, the time and message complexity for
      finding the minimum was tested as a sanity check on the implementation consistency.
      The graphs (see Figure~\ref{fig:mincplx}) are similar to those charted for finding
      the max with the exception of a bespoke \texttt{nkmin} program that was not created.
    }
    \paragraph{}{
      In future implementations of \texttt{nkithr}, it might be worth the added
      code complexity to add case logic for the minimum and maximum cases that employ
      the more specialized algorithms since finding the extrema can be a frequent use
      case of the more general $i$th selection in practice.
    }
  }

  \subsection{Finding the Median}{
    \paragraph{}{
      \begin{figure}
        \begin{tikzpicture}
          \footnotesize
          \begin{groupplot}[
              group style={
                group size=4 by 4,
                xlabels at=edge bottom,
                ylabels at=edge left,
                vertical sep=0.5in
              },
              height=1.5in,
              width=1.5in,
              xlabel=$n$,
              ylabel=$t$ (ns),
              tick label style={font=\tiny},
              label style={font=\tiny},
              legend style={font=\tiny},
              legend entries={,$k = 1$,,$k = 2$,,$k = 4$,,$k = 8$,,$k = 16$,,$k = 32$},
              legend to name=nprocs
            ]
            % nkith
            \nextgroupplot[group/empty plot]
            \linregplots{Total Time}{./results/nkith/nkith}{.med}{n}{t}
            \linregplots{Messages}{./results/nkith/nkith}{.med}{n}{m}
            \linregplots{Rounds}{./results/nkith/nkith}{.med}{n}{r}
            % nkithr (Knuth-B)x
            \linregplots{Exchange Time}{./results/nkithr-knuthb/nkithr}{.med}{n}{t}
            \linregplots{}{./results/nkithr-knuthb/nkithr}{.med}{n}{T}
            \linregplots{}{./results/nkithr-knuthb/nkithr}{.med}{n}{m}
            \linregplots{}{./results/nkithr-knuthb/nkithr}{.med}{n}{r}
            % nkithr (Standard)
            \linregplots{}{./results/nkithr-stdrnd/nkithr}{.med}{n}{t}
            \linregplots{}{./results/nkithr-stdrnd/nkithr}{.med}{n}{T}
            \linregplots{}{./results/nkithr-stdrnd/nkithr}{.med}{n}{m}
            \linregplots{}{./results/nkithr-stdrnd/nkithr}{.med}{n}{r}
            % nkithr -D
            \linregplots{}{./results/nkithr-d/nkithr-d}{.med}{n}{t}
            \linregplots{}{./results/nkithr-d/nkithr-d}{.med}{n}{T}
            \linregplots{}{./results/nkithr-d/nkithr-d}{.med}{n}{m}
            \linregplots{}{./results/nkithr-d/nkithr-d}{.med}{n}{r}
          \end{groupplot}
          \node at (group c1r1) {\pgfplotslegendfromname{nprocs}};
          \node[anchor=south] at ($(group c2r1.north east)!0.5!(group c3r1.north west)$){\texttt{nkith}};
          \node[anchor=south] at ($(group c2r2.north east)!0.5!(group c3r2.north west)$){\texttt{nkithr} (Knuth-B)};
          \node[anchor=south] at ($(group c2r3.north east)!0.5!(group c3r3.north west)$){\texttt{nkithr} (Default)};
          \node[anchor=south] at ($(group c2r4.north east)!0.5!(group c3r4.north west)$){\texttt{nkithr} (Deterministic)};
        \end{tikzpicture}
        \caption{Empirical Complexity of Finding the Median}
        \label{fig:mediancplx}
      \end{figure}
    }
    \paragraph{}{
      The complexities observed in finding the median (see Figure~\ref{fig:mediancplx})
      are again similar to those seen for the maximum and minimum, but with some striking
      differences. The time complexity for \texttt{nkithr} to find the median
      was longer by about half in terms of wall-clock time and rounds.
      The pivot selection was notably worse in the deterministic \texttt{nkithr}
      case, at least in terms of variance, if not mean, over the rounds required.
      In retrospect, a better deterministic selection would have been made
      by simply choosing the candidate with the maximum pre-normalization weight
      instead of the first non-zero weight.
    }
    \paragraph{}{
      Some of this behavior may be explained by the fact that random selection
      leading to the extrema heavily favors reduction from one side of the
      equation across all nodes, and so larger fractions of the population
      are taken out of consideration in each iteration. With the median,
      the side on which the reduction occurs may oscillate slowly until
      convergence.
    }
  }
}

\section{Conclusions}{
  \paragraph{}{
    The inherent power of algorithm randomization is aptly demonstrated in the
    trial results. Choosing a deterministic algorithm is not difficult, but
    in almost all cases, the random versions outperformed the implementations
    which used deterministic methods. It would appear that even the potentially
    best (as yet unimplemented) deterministic pivoting method simply copies the
    behavior of the randomized pivot selection.
  }
  \paragraph{}{
    The results also serve as a clear evidence that no amount of ``cleverness''
    in implementation can overcome the tyranny of scale. While familiarity with
    implementation details improved the performance of the randomized algorithm
    over more naive possible implementations of the same algorithm, it needed
    the solid algorithmic found upon which it could build.
  }
  \paragraph{}{
    The congress of ideas between the deterministic and randomized subfields
    of distributed algorithms is best kept free and fluid. Whether it is a
    deterministic algorithm informed by a randomized experiment or a recognition
    of a common special case that can be more rapidly decomposed in a deterministic
    fashion than the unbiased random approach would allow, the field as a whole
    will be richer for it.
  }
}

\printbibliography
\end{document}
