\documentclass[11pt,epsf]{article}
\usepackage{amssymb,amsmath,amsthm,amsfonts,mathrsfs,color}
\usepackage{epsfig}
\usepackage{latexsym}
\usepackage{verbatim}
\usepackage{setspace}
\usepackage{algorithm}
\usepackage[noend]{algorithmic}
\usepackage{algorithmicext}
\usepackage{ifthen}
\usepackage{graphicx}
\usepackage{url}
\usepackage{hyperref}
\usepackage[utf8]{luainputenc}
\usepackage[bibencoding=utf8,backend=biber]{biblatex}
\addbibresource{cosc6326-pa1-michael-yantosca.bib}
\usepackage{fancyhdr}
\pagestyle{fancy}
\lhead{{\footnotesize{COSC6326 PA 1}}}
\rhead{{\footnotesize{Michael Yantosca}}}

\usepackage{longtable}
\usepackage{pgfplots}
\usepackage{pgfplotstable}
\usepgfplotslibrary{external}
\usepgfplotslibrary{statistics}
\usepgfplotslibrary{groupplots}
\usetikzlibrary{pgfplots.groupplots, external}
\tikzexternalize[]
\pgfplotsset{
  tick label style={font=\footnotesize},
  label style={font=\small},
  legend style={font=\small},
  compat=newest
}
\pgfplotstableset{
  col sep=comma,
  begin table=\begin{longtable},
  end table=\end{longtable},
  every head row/.append style={after row=\endhead}
}

\newtheorem{fact}{Fact}
\newtheorem{theorem}{Theorem}
\newtheorem{lemma}{Lemma}
\newtheorem{claim}{Claim}
\newtheorem{remark}{Remark}
\newtheorem{definition}{Definition}
\newtheorem{corollary}{Corollary}
\newtheorem{proposition}{Proposition}
\newtheorem{example}{Example}
\newtheorem{observation}{Observation}
\newtheorem{exercise}{Exercise}
\newtheorem{statement}{Statement}
\newtheorem{problem}{Problem}

\newcommand{\TODO}[0]{\textbf{\color{red}{TODO}}}

% \linregplots{title}{prefix}{suffix}{x}{y}
\newcommand{\linregplots}[5]{
  \nextgroupplot[title=#1]
  \addplot [red, only marks, mark size=0.5] table [x=#4, y=#5] {#2.k1#3.log};
  \addplot [red, no markers] table [x=#4,y={create col/linear regression={y=#5}}] {#2.k1#3.log};
  \addplot [blue, only marks, mark size=0.5] table [x=#4, y=#5] {#2.k2#3.log};
  \addplot [blue, no markers] table [x=#4,y={create col/linear regression={y=#5}}] {#2.k2#3.log};
  \addplot [green, only marks, mark size=0.5] table [x=#4, y=#5] {#2.k4#3.log};
  \addplot [green, no markers] table [x=#4,y={create col/linear regression={y=#5}}] {#2.k4#3.log};
  \addplot [orange, only marks, mark size=0.5] table [x=#4, y=#5] {#2.k8#3.log};
  \addplot [orange, no markers] table [x=#4,y={create col/linear regression={y=#5}}] {#2.k8#3.log};
  \addplot [purple, only marks, mark size=0.5] table [x=#4, y=#5] {#2.k16#3.log};
  \addplot [purple, no markers] table [x=#4,y={create col/linear regression={y=#5}}] {#2.k16#3.log};
  \addplot [brown, only marks, mark size=0.5] table [x=#4, y=#5] {#2.k32#3.log};
  \addplot [brown, no markers] table [x=#4,y={create col/linear regression={y=#5}}] {#2.k32#3.log};
}

\date{}
\title{COSC6326 Programming Assignment 1}
\author{Michael Yantosca}
\begin{document}
\maketitle
\tableofcontents

\section{Introduction}{
  \paragraph{}{
    As problem sizes continue to scale, it would behoove algorithmic researchers to explore
    not only more distributed and decentralized approaches but especially randomized versions
    of these approaches. One particular information retrieval problem, namely the selection
    of the $i$th element among a large set of $n$ inherently ordered elements provides a basic
    but effective overview of some of the challenges at hand and the benefits that a randomized,
    distributed approach provides along with some of the insights that implementing such a
    system may offer, not only in the field of randomized algorithms, but in the field of
    deterministic algorithms, as well.
  }
}

\section{Analysis}{
  \subsection{\texttt{nkith}}{
    \paragraph{}{
      \begin{algorithm}
        \footnotesize
        \caption{\textsc{MP-Ith-Select}, Distributed Algorithm for Selecting the \emph{i}th Element of \emph{n} Numbers}
        \begin{algorithmic}
          \label{alg:nkith}
          \REQUIRE{$G$, a PRNG}\autocite{mt19337}
          \REQUIRE{$n$, global population size}
          \REQUIRE{$r$, node rank}
          \REQUIRE{$k$, number of nodes}
          \REQUIRE{
            \[
              m \gets n \div k +
              \begin{cases}
                1, & \text{ if } n \text{ mod } k > r \\
                0, & \text{otherwise}
              \end{cases},
              \text{ local population size}
            \]
          }
          \REQUIRE{$p \gets 0$, round-robin pivot selection counter}
          \REQUIRE{$b_l \gets 0$, live local population lower bound}
          \REQUIRE{$b_u \gets m$, live local population upper bound}
          \REQUIRE{$|S_{1,r}| \gets 0$, local cardinality counter for elements below the pivot}
          \REQUIRE{$|P_r| \gets 0$, local cardinality counter for elements less than or equal to the pivot}
          \REQUIRE{$|S_1| \gets 0$, global cardinality counter for elements below the pivot}
          \REQUIRE{$|P| \gets 0$, global cardinality counter for elements less than or equal to the pivot}
          \STATE{Seed $G$ with the current epoch time.}
          \FOR{$j = 0 \text{ to } m - 1$}{
            \STATE{$M[j] = \textsc{Generate}(G)$}
          }\ENDFOR
          \STATE{\textsc{Qsort}($M$)\autocite{qsort}}
          \WHILE{\neg ($|S_1| < i$ and $|S_1| + |P| \geq i$)}{
            \STATE{$c_r \gets M[b_l + (b_u - b_l) / 2]$}
            \STATE{All nodes send $c_r$ to the root node.}
            \IF{$r = 0$}{
              \STATE{$c = c_p$}
              \STATE{$p = (p + 1) \text{ mod } k$}
              \STATE{Broadcast $c$ to all nodes.}
            }\ENDIF
            \STATE{$|S_{1,r}| \gets b_l$}
            \WHILE{$M[|S_{1,r}|] < c$ and $|S_1| < b_u$}{
              \STATE{$|S_{1,r}| = |S_{1,r}| + 1$}
            }\ENDWHILE
            \STATE{$|P_r| \gets |S_{1,r}|$}
            \WHILE{$M[|P_r|] == c$ and $|P_r| < b_u$}{
              \STATE{$|P_r| = |P_r| + 1$}
            }\ENDWHILE
            \STATE{All nodes send $|S_{1,r}|$ and $|P_r|$ to the root node.}
            \IF{$r = 0$}{
              \STATE{$|S_1| \gets \sum_{r = 0}^{k-1} |S_{1,r}|$}
              \STATE{$|P| \gets \sum_{r = 0}^{k-1} |P_r|$}
              \STATE{Broadcast $|S_1|$ and $|P|$ to all nodes.}
            }\ENDIF
            \IF{$|S_1| > i - 1$}{
              \STATE{$b_u \gets |S_{1,r}|$}
            }\ELSIF{$|P| < i$}{
              \STATE{$b_l \gets |P_r|$}
            }\ENDIF
          }\ENDWHILE
          \STATE{The root node announces $c$ as the $i$th element.}
        \end{algorithmic}
      \end{algorithm}
    }
    \paragraph{}{
      A full analysis of Algorithm~\ref{alg:nkith} is omitted since it does not faithfully
      reproduce the sequential algorithm. Indeed, a proper analysis is beyond the scope of
      this assignment since the implementation of \textsc{Qsort} is dependent on the
      particular POSIX implementation on which the code is run. The pseudocode is included
      for reference and completeness since this first attempt served as the progenitor for
      the correct implementation in Algorithm~\ref{alg:nkithr}.
    }
  }

  \subsection{\texttt{nkithr}}{
    \paragraph{}{
      \begin{algorithm}
        \footnotesize
        \caption{\textsc{Random-Pivot-Select}, Distributed Algorithm for Selecting a Pivot Candidate}
        \begin{algorithmic}
          \label{alg:random-pivot-select}
          \REQUIRE{$k$, number of nodes}
          \REQUIRE{$b_l$, live local population lower bound}
          \REQUIRE{$b_u$, live local population upper bound}
          \STATE{$G_{cand} \gets$ uniform distribution over $[b_l, b_u)$}\autocite{uniformintdist}
          \STATE{$c_r \gets M[\textsc{Generate}(G_{cand})]$}
          \STATE{$w_r \gets b_u - b_l$}
          \STATE{All nodes send $c_r$ and $w_r$ to the root node.}
          \IF{$r = 0$}{
            \STATE{$w \gets \sum_{r=0}^{k-1} w_r$}
            \STATE{$G_{pivot} \gets$ distribution over $[0,k)$ weighted by $w_r/w$.}\autocite{discretedist}
            \STATE{$p = \textsc{Generate}(G_{pivot})$}
            \STATE{$c = c_p$}
            \STATE{Broadcast $c$ to all nodes.}
          }\ENDIF
        \end{algorithmic}
      \end{algorithm}
    }

    \paragraph{}{
      \begin{algorithm}
        \footnotesize
        \caption{\textsc{MP-Ith-Select-Revised}, Distributed Algorithm for Selecting the \emph{i}th Element of \emph{n} Numbers}
        \begin{algorithmic}
          \label{alg:nkithr}
          \REQUIRE{$G_{popl}$, a population PRNG}\autocite{mt19337}
          \REQUIRE{$n$, global population size}
          \REQUIRE{$r$, node rank}
          \REQUIRE{$k$, number of nodes}
          \REQUIRE{$c$, the current pivot}
          \REQUIRE{
            \[
              m \gets n \div k +
              \begin{cases}
                1, & \text{ if } n \text{ mod } k > r \\
                0, & \text{otherwise}
              \end{cases},
              \text{ local population size}
            \]
          }
          \REQUIRE{$b_l \gets 0$, live local population lower bound}
          \REQUIRE{$b_u \gets m$, live local population upper bound}
          \REQUIRE{$|S_1| \gets 0$, global cardinality for elements below the pivot}
          \REQUIRE{$|P| \gets 0$, global cardinality for elements less than or equal to the pivot}
          \STATE{Seed $G_{popl}$ with the current epoch time.}
          \FOR{$j = 0 \text{ to } m - 1$}{
            \STATE{$M[j] = \textsc{Generate}(G_{popl})$}
          }\ENDFOR
          \WHILE{\neg ($|S_1| < i$ and $|S_1| + |P| \geq i$)}{
            \STATE{$\textsc{Random-Pivot-Select}(b_l, b_u)$ \COMMENT{Algorithm~\ref{alg:random-pivot-select}}}
            \STATE{$s_1 \gets b_l$}
            \STATE{$s_p \gets 0$}
            \STATE{$s_2 \gets b_u$}
            \STATE{$j \gets b_l$}
            \FOR{$j = b_l \text{ to } b_u - 1$}{
              \IF{$M[j] < c$}{
                \STATE{$Q[s_1] \gets M[j]$}
                \STATE{$s_1 = s_1 + 1$}
              }\ELSIF{$M[j] > c$}{
                \STATE{$s_2 = s_2 -1$}
                \STATE{$Q[s_2] \gets M[j]$}
              }\ELSE{
                \STATE{$s_p \gets s_p + 1$}
              }\ENDIF
            }\ENDFOR
            \STATE{$s_p \gets s_p + s_1$}
            \FOR{$j = s_1 \text{ to } s_p$}{
              \STATE{$Q[j] \gets c$}
            }\ENDFOR
            \IF{$b_l < b_u$}{
              \STATE{Copy $Q[b_l, b_u-1]$ to $M[b_l, b_u-1]$.}
            }\ENDIF
            \STATE{All nodes send $s_1$ and $s_p$ to the root node.}
            \IF{$r = 0$}{
              \STATE{$|S_1| \gets \sum_{r = 0}^{k-1} s_{1,r}$}
              \STATE{$|P| \gets \sum_{r = 0}^{k-1} s_{p,r}$}
              \STATE{Broadcast $|S_1|$ and $|P|$ to all nodes.}
            }\ENDIF
            \IF{$|S_1| > i - 1$}{
              \STATE{$b_u \gets s_1$}
            }\ELSIF{$|P| < i$}{
              \STATE{$b_l \gets s_p$}
            }\ENDIF
          }\ENDWHILE
          \STATE{The root node announces $c$ as the $i$th element.}
        \end{algorithmic}
      \end{algorithm}
    }
    \begin{theorem}
      \label{thm:nkithr}
      Algorithm~\ref{alg:nkithr} determines the $i$th element of a set of $n$ numbers
      distributed over $k$ nodes with time complexity $O(\log n)$ rounds and message complexity
      $O(k \log n)$ on average.
    \end{theorem}
    \begin{lemma}
      \label{lem:nkithr-t}
      Algorithm~\ref{alg:nkithr} determines the $i$th element of a set of $n$ numbers
      distributed over $k$ nodes with time complexity $O(\log n)$ rounds on average.
    \end{lemma}
    \begin{proof}
      The proof follows the sketch outlined by the proof for the time
      complexity of Randomized Quicksort given in the sequential algorithms
      text by Pandurangan\autocite[168]{ALG}. For simplicity, we limit the proof to the
      case where the $i$th element is the median, i.e., $\lfloor \frac{n}{2} \rfloor$
      among $n$ numbers and the pivot selection is random and independent of the
      population distribution.
      \paragraph{}{
        The first order of business is to simplify the proof so that we can
        speak in global terms without having to perform complicated
        if not intractable calculations on local probabilities.
      }
      \begin{lemma}
        \label{lem:goodpivot}
        The pivot selected by Algorithm~\ref{alg:random-pivot-select} chooses any
        member of the live (i.e., considered) population $n_j$ in the $j$th iteration
        with uniform probability $\frac{1}{n_j}$.
      \end{lemma}
      \begin{proof}
        The local candidate is selected from the local live population of size $m_j = b_{u,j} - b_{l,j}$
        with $\frac{1}{m_j}$ probability. The corresponding weight sent to the root
        node with the candidate is $m_j$. The global live population is faithfully
        recorded at the root as $n_j = \sum_{h=0}^{k-1} m_{h,j}$. The weights are normalized
        against $n_j$ so that each becomes respectively $\frac{m_{h,j}}{n_j}$.
        The node identifier is selected according to a discrete distribution weighted
        with the aforementioned weights. The probability that any given element
        will be selected as the pivot is given as follows:
        \begin{align*}
          P(\text{element selected globally})
          &= P(\text{node selected globally}|\text{element selected locally})\\
          &\phantom{= }\cdot P(\text{element selected locally}) \\
          &= \frac{m_{h,j}}{n_j} \cdot \frac{1}{m_{h,j}} \\
          &= \frac{1}{n_j}
        \end{align*}
        Thus, Algorithm~\ref{alg:random-pivot-select} selects each pivot in each
        iteration from among the live element population with uniform probability.
      \end{proof}
      \paragraph{}{
        By Lemma~\ref{lem:goodpivot}, we can consider the partitioning of the population
        in a global sense so long as the implementation faithfully reports the cardinalities
        of the subsets of $S_1$ (less than the pivot) and $P$ (equal to the pivot).
        We know that this is the case since the element-wise comparisons done locally
        appropriately bin the live local population into the respective subsets,
        and the cardinalities of the subsets are summed at the root and rebroadcast
        to all nodes for the purpose of updating the live population boundaries.
      }
      \paragraph{}{
        As with the proof for Theorem 8.8\autocite[168-169]{ALG}, let us consider then
        the division of outcomes into good and bad sets. Let the good outcome be that
        where the pivot is chosen in the middle third of the live global population, i.e.,
        that neither $|S_1|$ nor $|S_2|$ (greater than the pivot) exceeds $\frac{2n_j}{3}$,
        and let the bad outcome be the converse.
      }
      \paragraph{}{
        We cannot have more than $\log_{3/2} n$ good outcomes, or else the population will
        have been exhausted by the constant reduction by $\frac{2}{3}$. , the expectation for execution time is $3c\log n$,
        where $c$ is a fixed constant independent of the problem size such that
        $\log_{3/2} n < c \log n$.
      }
      \paragraph{}{
        If we consider longer paths of execution that have at most $c \log n$ good outcomes
        but are longer than that, i.e., paths of length $ac \log n$ for some $a > 1$, we can
        use the Chernoff Lower Tail Bound\autocite[323]{ALG} to provide a probabilistic
        limit on the execution time in terms of number of iterations.
        Since good outcomes occur with probability $\frac{1}{3}$, the expectation of good
        outcomes on a path of length $ac \log n$ would be $\frac{1}{3}ac \log n$.
        By choosing $\delta = 3/a$, we get a probability $P(X_{good} < c \log n) \leq \frac{1}{n^2}$.
        Consequently, with $n$ elements in play at the start, the single
        path probability becomes union bounded as $1/n$\autocite[169]{ALG}.
      }
      \paragraph{}{
        Thus, the number of iterations is on the order of $O(\log n)$ rounds with
        high probability since the greatest possible probability to exceed $ac \log n$
        rounds is $1/n$.
      }
    \end{proof}
    \begin{lemma}
      \label{lem:nkithr-m}
      Algorithm~\ref{alg:nkithr} determines the $i$th element of a set of $n$ numbers
      distributed over $k$ nodes with message complexity $O(k \log n)$ on average.
    \end{lemma}
    \begin{proof}
      By Lemma~\ref{lem:nkithr-t}, we know that Algorithm~\ref{alg:nkithr} takes
      $O(\log n)$ iterations on average. Within each iteration, the following message exchanges occur:
      \begin{enumerate}
      \item{$k - 1$ nodes send pivot candidates to the root.}
      \item{$k - 1$ nodes send pivot weights to the root.}
      \item{The root sends the pivot-elect to $k - 1$ nodes.}
      \item{$k - 1$ nodes send local cardinalities $s_1$ and $s_p$ to the root.}
      \item{The root sends global cardinalities $|S_1|$ and $|S_1| + |P|$ to $k - 1$ nodes.}
      \end{enumerate}
      With 5 exchanges of $k - 1$ messages, each iteration produces $5k - 5$. Even if we are
      nitpicky and account the 2 element local cardinality messages as 2 exchanges, we still
      only have $6k - 6$, which in either case is $O(k)$ messages per iteration. Multiplying
      by $O(\log n)$ iterations, we arrive at total message complexity of $O(k \log n)$,
      which was to be shown.
    \end{proof}
    \begin{proof}
      By Lemma~\ref{lem:nkithr-t} and Lemma~\ref{lem:nkithr-m}, we see that the two
      complexities are satisfied. Hence, Theorem~\ref{thm:nkithr} holds.
    \end{proof}
  }
}

\section{Results}{
  \paragraph{}{
    \begin{figure}
      \begin{tikzpicture}
        \footnotesize
        \begin{groupplot}[
            group style={
              group size=4 by 5,
              xlabels at=edge bottom,
              ylabels at=edge left
            },
            height=1.75in,
            width=1.75in,
            xlabel=$n$,
            ylabel=$t$ (ns),
            tick label style={font=\tiny},
            label style={font=\tiny},
            legend style={font=\tiny},
            legend entries={,$k = 1$,,$k = 2$,,$k = 4$,,$k = 8$,,$k = 16$,,$k = 32$},
            legend to name=nprocs
          ]
          % nkith
          \nextgroupplot[group/empty plot]
          \linregplots{\footnotesize Total Time}{./results/nkith/nkith}{.max}{n}{t}
          \linregplots{\footnotesize Messages}{./results/nkith/nkith}{.max}{n}{m}
          \linregplots{\footnotesize Rounds}{./results/nkith/nkith}{.max}{n}{r}
          % nkithr (Knuth-B)
          \linregplots{\footnotesize Exchange Time}{./results/nkithr-knuthb/nkithr}{.max}{n}{t}
          \linregplots{}{./results/nkithr-knuthb/nkithr}{.max}{n}{T}
          \linregplots{}{./results/nkithr-knuthb/nkithr}{.max}{n}{m}
          \linregplots{}{./results/nkithr-knuthb/nkithr}{.max}{n}{r}
          % nkithr (Standard)
          \linregplots{}{./results/nkithr-stdrnd/nkithr}{.max}{n}{t}
          \linregplots{}{./results/nkithr-stdrnd/nkithr}{.max}{n}{T}
          \linregplots{}{./results/nkithr-stdrnd/nkithr}{.max}{n}{m}
          \linregplots{}{./results/nkithr-stdrnd/nkithr}{.max}{n}{r}
          % nkithr -D
          \linregplots{}{./results/nkithr-d/nkithr-d}{.max}{n}{t}
          \linregplots{}{./results/nkithr-d/nkithr-d}{.max}{n}{T}
          \linregplots{}{./results/nkithr-d/nkithr-d}{.max}{n}{m}
          \linregplots{}{./results/nkithr-d/nkithr-d}{.max}{n}{r}
          % nkmax
          \nextgroupplot[group/empty plot]
          \linregplots{}{./results/nkmax/nkmax}{}{n}{t}
          \linregplots{}{./results/nkmax/nkmax}{}{n}{m}
          \linregplots{}{./results/nkmax/nkmax}{}{n}{r}
        \end{groupplot}
        \node at (group c1r1) {\pgfplotslegendfromname{nprocs}};
      \end{tikzpicture}
      \caption{Empirical Complexity of Finding the Max}
      \label{fig:maxcplx}
    \end{figure}
  }

  \paragraph{}{
    \begin{figure}
      \begin{tikzpicture}
        \footnotesize
        \begin{groupplot}[
            group style={
              group size=4 by 4,
              xlabels at=edge bottom,
              ylabels at=edge left
            },
            height=1.75in,
            width=1.75in,
            xlabel=$n$,
            ylabel=$t$ (ns),
            tick label style={font=\tiny},
            label style={font=\tiny},
            legend style={font=\tiny},
            legend entries={,$k = 1$,,$k = 2$,,$k = 4$,,$k = 8$,,$k = 16$,,$k = 32$},
            legend to name=nprocs
          ]
          % nkith
          \nextgroupplot[group/empty plot]
          \linregplots{Total Time}{./results/nkith/nkith}{.min}{n}{t}
          \linregplots{Messages}{./results/nkith/nkith}{.min}{n}{m}
          \linregplots{Rounds}{./results/nkith/nkith}{.min}{n}{r}
          % nkithr (Knuth-B)
          \linregplots{Exchange Time}{./results/nkithr-knuthb/nkithr}{.min}{n}{t}
          \linregplots{}{./results/nkithr-knuthb/nkithr}{.min}{n}{T}
          \linregplots{}{./results/nkithr-knuthb/nkithr}{.min}{n}{m}
          \linregplots{}{./results/nkithr-knuthb/nkithr}{.min}{n}{r}
          % nkithr (Standard)
          \linregplots{}{./results/nkithr-stdrnd/nkithr}{.min}{n}{t}
          \linregplots{}{./results/nkithr-stdrnd/nkithr}{.min}{n}{T}
          \linregplots{}{./results/nkithr-stdrnd/nkithr}{.min}{n}{m}
          \linregplots{}{./results/nkithr-stdrnd/nkithr}{.min}{n}{r}
          % nkithr -D
          \linregplots{}{./results/nkithr-d/nkithr-d}{.min}{n}{t}
          \linregplots{}{./results/nkithr-d/nkithr-d}{.min}{n}{T}
          \linregplots{}{./results/nkithr-d/nkithr-d}{.min}{n}{m}
          \linregplots{}{./results/nkithr-d/nkithr-d}{.min}{n}{r}
        \end{groupplot}
        \node at (group c1r1) {\pgfplotslegendfromname{nprocs}};
      \end{tikzpicture}
      \caption{Empirical Complexity of Finding the Min}
      \label{fig:mincplx}
    \end{figure}
  }

  \paragraph{}{
    \begin{figure}
      \begin{tikzpicture}
        \footnotesize
        \begin{groupplot}[
            group style={
              group size=4 by 4,
              xlabels at=edge bottom,
              ylabels at=edge left
            },
            height=1.75in,
            width=1.75in,
            xlabel=$n$,
            ylabel=$t$ (ns),
            tick label style={font=\tiny},
            label style={font=\tiny},
            legend style={font=\tiny},
            legend entries={,$k = 1$,,$k = 2$,,$k = 4$,,$k = 8$,,$k = 16$,,$k = 32$},
            legend to name=nprocs
          ]
          % nkith
          \nextgroupplot[group/empty plot]
          \linregplots{Total Time}{./results/nkith/nkith}{.med}{n}{t}
          \linregplots{Messages}{./results/nkith/nkith}{.med}{n}{m}
          \linregplots{Rounds}{./results/nkith/nkith}{.med}{n}{r}
          % nkithr (Knuth-B)
          \linregplots{Exchange Time}{./results/nkithr-knuthb/nkithr}{.med}{n}{t}
          \linregplots{}{./results/nkithr-knuthb/nkithr}{.med}{n}{T}
          \linregplots{}{./results/nkithr-knuthb/nkithr}{.med}{n}{m}
          \linregplots{}{./results/nkithr-knuthb/nkithr}{.med}{n}{r}
          % nkithr (Standard)
          \linregplots{}{./results/nkithr-stdrnd/nkithr}{.med}{n}{t}
          \linregplots{}{./results/nkithr-stdrnd/nkithr}{.med}{n}{T}
          \linregplots{}{./results/nkithr-stdrnd/nkithr}{.med}{n}{m}
          \linregplots{}{./results/nkithr-stdrnd/nkithr}{.med}{n}{r}
          % nkithr -D
          \linregplots{}{./results/nkithr-d/nkithr-d}{.med}{n}{t}
          \linregplots{}{./results/nkithr-d/nkithr-d}{.med}{n}{T}
          \linregplots{}{./results/nkithr-d/nkithr-d}{.med}{n}{m}
          \linregplots{}{./results/nkithr-d/nkithr-d}{.med}{n}{r}
        \end{groupplot}
        \node at (group c1r1) {\pgfplotslegendfromname{nprocs}};
      \end{tikzpicture}
      \caption{Empirical Complexity of Finding the Median}
      \label{fig:mediancplx}
    \end{figure}
  }

  \section{Conclusions}{

  }
}

\printbibliography
\end{document}
